% !TEX encoding = UTF-8
% !TEX program = pdflatex
% !TeX spellcheck = it_IT

\documentclass[italian,a4paper]{europasscv}
\usepackage[italian]{babel}

\usepackage[backend=biber,autolang=hyphen,sorting=none,style=numeric,maxbibnames=99,doi=false,isbn=false,maxcitenames=2]{biblatex}
\usepackage{csquotes}
\usepackage{europasscv-bibliography}
\usepackage[a-1b]{pdfx} %for a PDF/A document

\bibliography{europasscv_stecca_recent}
% in the bibliography, mark all occurrences in bold:
\ecvbibhighlight{Stecca}{Giuseppe}{G.}

\ecvname{Giuseppe Stecca}
\ecvaddress{Via Franco Sacchetti 104, 00137 Roma}
\ecvmobile{06 338 17 94 305}
%\ecvtelephone{06 871 40 665}
%\ecvworkphone{06 4993 7127}
\ecvemail{giuseppe.stecca@gmail.com, giuseppe.stecca@iasi.cnr.it}
\ecvhomepage{www.iasi.cnr.it}
\ecvhomepage{https://scholar.google.it/citations?user=QDZ4kJoAAAAJ}
% \ecvgithubpage{www.github.com/smith}
% \ecvgitlabpage{www.gitlab.com/smith}
% \ecvlinkedinpage{www.linkedin.com/in/katie-smith}
\ecvorcid[label, link]{0000-0001-5876-4538}
%\ecvim{Skype}{gius237}
%\ecvim{Google Talk}{ksmith}

\ecvgender{Maschio}
\ecvdateofbirth{anonymized} %23/12/1974}
\ecvnationality{Italiana}

\ecvpicture[width=3cm]{cv2025.jpg}

%\date{}

\begin{document}
  \begin{europasscv}

  \ecvpersonalinfo
  \ecvbigitem{Profilo}{Primo Ricercatore }
	\ecvitem{} {Giuseppe Stecca si è laureato in Ingegneria Gestionale e ha ottenuto il Dottorato di Ricerca in Ingegneria Economico Gestionale presso l'Università di Roma ``Tor Vergata'' nel 2000 e nel 2004 rispettivamente. Presso la stessa Università ha conseguito un master II Livello in Organizzazione e Gestione dell'Impresa.  Ha una docenza in ``Supply Chain Management'', all'Università di Roma ``Tor Vergata'', e docente di ``Modelli di Sistemi di Produzione'' all'Università ``Guglielmo Marconi''. E' primo ricercatore al Consiglio Nazionale delle Ricerche, Istituto di Analisi dei Sistemi ed Informatica ``Antonio Ruberti'' - CNR-IASI. Ha un'esperienza pluriennale in progetti di ricerca regionali, nazionali e internazionali. E' valutatore tecnico scientifico, per il MiSE. Ha avuto diverse esperienze all'estero come visiting professor. I suoi principali interessi di ricerca riguardano modellazione matematica, simulazione e algoritmi di ottimizzazione di sistemi di trasporto, logistica, servizi e sistemi di produzione. Le principali applicazioni studiate riguardano green supply chain management, e digital twin. E' coautore di più di 50 pubblicazioni. E' iscritto all'ordine degli Ingegneri della Provincia di Roma (area civile, ambientale, industriale e informatica).}
	
\ecvsection{Educazione}

	\ecvtitlelevel{2002 -- 2004}{Dottorato di Ricerca in Ingegneria Economico Gestionale}{ISCED~6}
	\ecvitem{}{Università di Roma ``Tor Vergata", conseguito il 14/12/2004, XVIII ciclo}
	
	\ecvtitle{2001 -- 2002}{Master II livello In Organizzazione e Gestione dell'Impresa}
	\ecvitem{}{Università di Roma ``Tor Vergata", conseguito il 13/12/2002, voto A}
	
	\ecvtitle{1993 -- 2000}{Laurea  (Vecchio Ordinamento) in Ingegneria Gestionale D.M. 509/99)}
	\ecvitem{}{Università di Roma ``Tor Vergata", conseguita il 07/12/2000, voto 98/100}
	
\ecvsection{Esperienze Professionali}
  
  \ecvtitle{09/2011 -- in corso}{Ricercatore}
  \ecvitem{}{Consiglio Nazionale delle Ricerche, Istituto di Analisi dei Sistemi ed Informatica ``Antonio Ruberti''\newline Via dei Taurini 19, 00185 Roma. Primo ricercatore da Gennaio 2023}
  \ecvitem{}{Modellazione e ottimizzazione di processi in logistica, manufacturing, trasporto, economia circolare, sostenibilità.}

  \ecvtitle{01/2001 -- 08/2011}{Titololare Assegno di Ricerca / Collaboratore di Ricerca}
  \ecvitem{}{Consiglio Nazionale delle Ricerche}
  \ecvitem{}{Modellazione e ottimizzazione di processi logistici, di manufacturing, trasporto, economia circolare e sostenibilità.}
  
 
\ecvtitle{2015 -- ora}{Coordinamento e rendicontazione di Progetto}
  \ecvitem{-}{    	Coordinatore per CNR-IASI del progetto di ricerca ARPA - Autonomous and flexible manufacturing and augmented reality techniques for processes automation (11/2022 - 12/2023). attività task 8.4: Models and systems for preventive maintenance and for the management of logistics risks. Project funded by PON MiSE n. F/190046/01-03/X44 id. 46.}
  \ecvitem{-}{Coordinatore per CNR-IASI del progetto PNRR CN MOST Centro per la Mobilità Sostenibile, Spoke 7 CCAM (Cooperative Connected Autonomous Vehicles), by 01/09/2022.}
   \ecvitem{-}{ Coordinatore per CNR-IASI  e responsabile scientifico per il progetto ``PIPER - Piattaforma Intelligente per l'Ottimizzazione delle operazioni di riciclo'' (Intelligent Platform for the Optimization of Recycling Operations)  POR FESR LAZIO 2014-2020 ``Progetti Gruppi di Ricerca 2020" by 15/04/2022.}
   
   \ecvitem{-}{  Coordinatore per CNR-IASI di ``CTEMT - Casa delle Tecnologie Emergenti di Matera'' (House of Emergent Technologies Matera), Task 1.2, finanziato dal Fondo Crescita Sostenibile del MiSE, per lo sviluppo di digital twin urbani, in relazione a ottimizzazione sostenibile per la mobilità e l'accessibilità.}
   
   \ecvitem{-}{ Coordinatore per CNR-IASI e leader scientifico  del progetto  ``REMIND - Reverse Manufacturing Innovation Decision System'', CUP  B86H18000160002, Programma Regione Lazio POR FERS ``Economia Circolare'' by 1/11/2018.}
   	% Rif. Attestazione di servizio firmata from Direttore CNR-IASI Giovanni Rinaldi il 02/03/2017 prot.CNR-IASI 228 del 02/03/2017
   	\ecvitem{-}{Coordinatore per CNR-IASI del progetto ``ZED\&L Zero Emissions Distribution \& Logistics'', da 1/09/2018 a 27/02/2019.}
   	% Rif. Attestazione di servizio firmata from Direttore CNR-IASI Giovanni Rinaldi il 02/03/2017 prot.CNR-IASI 228 del 02/03/2017
   	
	\ecvitem{-}{Coordinatore per CNR-IASI del progetto ``MIE - Mobilit\`a Intelligente
  Ecosostenibile'' (Smart Eco-sustainable Mobility), n. CTN01\_00034\_594122, cluster "Tecnologie per Smart Communities", by 1/12/2016.}
  %Rif. Attestazione di servizio firmata dal Direttore CNR-IASI Giovanni Rinaldi il 02/03/2017 prot.CNR-IASI 228 del 02/03/2017
	
	\ecvitem{-}{Coordinatore per CNR-IASI del progetto PRIN ``SPORT'' - Smart Port Terminals, n. 2015XAPRKF dal 12/10/2016.}
  %Rif. Attestazione di servizio firmata dal Direttore CNR-IASI Giovanni Rinaldi il 02/03/2017 prot.CNR-IASI 228 del 02/03/2017; Lettera del principal investigator Walter Ukovic.
	\ecvitem{-}{Coordinatore per CNR-IASI del progetto ``SIGMA - Sistema Integrato di sensori In ambiente cloud per la Gestione Multirischio Avanzata'' (Integrated Sensor System in Cloud for Advanced Management of Multi Risks), n. PON01\_00683 by 09/06/2013.}
  %Rif. Attestazione di servizio firmata dal Direttore CNR-IASI Giovanni Rinaldi il 02/03/2017 prot.CNR-IASI 228 del 02/03/2017.
	\ecvitem{-}{ Coordinatore del progetto premiale  ``Tecnologie e sistemi innovativi per   la fabbrica del futuro e made in Italy'' (Technologies and Innovative Systems for the Factory of the Future), n. B82I12000420005, anni 2014 e 2015.}
  %Rif. Attestazione di servizio firmata dal Direttore CNR-IASI Giovanni Rinaldi il 02/03/2017 prot.CNR-IASI 228 del 02/03/2017
	\ecvitem{-}{ Coordinatore per CNR-IASI Attivit\`a Progettuali (AP) Fabbrica del Futuro (Activities for Factory of the Future), anno 2014.} %Rif. Attestazione di servizio firmata dal Direttore CNR-IASI Giovanni Rinaldi il 02/03/2017 prot.CNR-IASI 228 del 02/03/2017



\ecvtitle{01/09/2018 -- ora}{Valutatore Scientifico}
% \ecvitem{}{European Commission, Youth Unit, DG Education and Culture \newline 200, Rue de la Loi, 1049 Brussels (Belgium)}

	\ecvitem{-}{Valutatore ex-ante, in itinere e ex-post bandi  ``Infrastrutture di ricerca per il trasferimento tecnologico” nell’ambito del PR FESR 2021/2027 - AZIONE 1.1.2. - AZIONE 1.6.1. ``Sostegno al trasferimento tecnologico tra mondo della ricerca e delle imprese lombarde”, della Regione Lombardia - Finlombarda per conto MIZAR Consulting. Periodo: Gennaio 2026 - in corso.}

	\ecvitem{-}{Regione Emilia Romagna, Componente esperto Nuclei di Valutazione e valutatore ex ante per il bando investimenti e ricerca per le tecnologie strategiche STEP (Strategic Technologies for Europe Platform), e di altre procedure assimilabili, gestiti dal Settore Innovazione Sostenibile, Imprese, Filiere Produttive, predisposto in attuazione del Programma regionale FESR 21/27. Periodo: Giugno 2025 - in corso.}
	
	\ecvitem{-}{Valutatore ex Ante bandi ``POR FESR 21-27" - AZIONE 1.6.1. ``Sviluppo delle tecnologie critiche nei	progetti di partenariato tra PMI e Grandi imprese”, della Regione Lombardia per conto MIZAR Consulting. Periodo: Agosto 2025 - Dicembre 2025.}
		
	\ecvitem{-}{Valutatore ex Ante bandi ``Collabora \& Innova" - Azione 1.1.3.Obiettivo specifico 1.1 del PR FESR 2021/2027 della Regione Lombardia Regione Lombardia per conto MIZAR Consulting. Periodo: Aprile 2025 - in corso.}
		
	\ecvitem{-}{Nucleo di Valutazione per Sviluppo Toscana - Regione Toscana e supporto professionale in riferimento alla validazione della metodologia volta alla individuazione delle operazioni STEP. Periodo: Gennaio 2025 - in corso.}
		
	\ecvitem{-}{Valutatore ex Ante bandi PNRR ``Bandi a cascata RAISE - Robotics and AI for Socio-economic Empowerment'' Spoke 4. Committente: Università di Genova. Periodo: 20 Luglio 2024 - 20 Settembre 2024. Link: https://www.raiseliguria.it/bandi/}
		
	\ecvitem{-}{Valutatore ex ante Sviluppo toscana Bandi R\&S 2023 – Bando n.1: Progetti Strategici di 	ricerca e sviluppo  - PR FESR	TOSCANA 2021 – 2027, AZIONE 1.1.4 Ricerca e sviluppo per le imprese anche in raggruppamento con organismi di ricerca - approvati rispettivamente con i Decreti Dirigenziali n. 27716 del 29/12/2023 e n. 27717 del 29/12/2023. Committente: Sviluppo Toscana. Periodo Luglio - Agosto 2024.} 
		%Link: https://www.regione.toscana.it/-/progetti-strategici-di-ricerca-e-sviluppo-per-grandi-imprese-1 
		
	\ecvitem{-}{Valutatore ex ante Sviluppo toscana Bandi RS 2023 – Bando n.2: Progetti di R\&S per MPMI e Midcap - PR FESR	TOSCANA 2021 – 2027, AZIONE 1.1.4 Ricerca e sviluppo per le imprese anche in raggruppamento con organismi di ricerca - approvati rispettivamente con i Decreti Dirigenziali n. 27716 del 29/12/2023 e n. 27717 del 29/12/2023. Committente: Sviluppo Toscana. Periodo Luglio - Agosto 2024.} 
		%Link: https://www.regione.toscana.it/-/progetti-di-ricerca-e-sviluppo-delle-mpmi-e-midcap
		
	\ecvitem{-}{Valutatore Ex ante PNRR ``Bandi a cascata RAISE - Robotics and AI for Socio-economic Empowerment'' Spoke 5 . Committente: IIT - Istituto Italiano di Tecnologie. Ottobre, 1 - 31, 2023. Link: https://www.raiseliguria.it/bandi/}
		
	\ecvitem{-}{Esperto Tecnico Scientifico (valutazione ex ante, in itinere, in loco, ex post) di progetti di ricerca per CNR/Ministero dello Sviluppo Economico (MiSE), Accordo Innovazione DM 31/12/2021 (Primo Bando) (162). Committente CNR/MiSE. Periodo: dal 13/05/2023.} 
		%Link: https://www.mimit.gov.it/it/incentivi/scoperta-imprenditoriale
		
	\ecvitem{-}{Esperto Tecnico Scientifico (valutazione ex ante, in itinere, in loco, ex post) di progetti di ricerca per CNR/Ministero dello Sviluppo Economico (MiSE), Accordo Innovazione DM 31/12/2021 (Primo Bando) (276). Committente CNR/MiSE. Periodo: dal 18/04/2023.} 
		%Link: https://www.mimit.gov.it/it/incentivi/accordi-per-l-innovazione-2
		
		
	\ecvitem{-}{Esperto Tecnico Scientifico (valutazione ex ante, in itinere, in loco, ex post) di progetti di ricerca per CNR/Ministero dello Sviluppo Economico (MiSE), accordi di innovazione D.M. 24/5/2017 (55). Committente CNR/MiSE. Periodo: dal 22/09/2020.} 
		%Link: https://www.mimit.gov.it/it/normativa/decreti-ministeriali/decreto-ministeriale-24-maggio-2017-accordi-per-l-innovazione
		
	\ecvitem{-}{Esperto Tecnico Scientifico (valutazione ex ante, in itinere, in loco, ex post) di progetti di ricerca per CNR/Ministero dello Sviluppo Economico (MiSE), Sportello Fabbrica Intelligente DM 5/3/2018 (44, 94). Committente CNR/MiSE. Periodo: dal 13/05/2019.} 
		%Link: https://www.mimit.gov.it/it/normativa/decreti-ministeriali/decreto-ministeriale-5-marzo-2018-bando-fabbrica-intelligente-agrifood-e-scienze-della-vita
		
	\ecvitem{-}{Esperto Tecnico Scientifico (valutazione ex ante, in itinere, in loco, ex post) D.M. 5 marzo 2018 e del successivo D.D. 27	settembre 2018 accordi innovazione e inerente al settore FABBRICA INTELLIGENTE (46). Committente CNR /Mise. Periodo: dal 15/12/2021.} 
		%Link: https://www.mimit.gov.it/index.php/it/incentivi/bando-fabbrica-intelligente-agrifood-e-scienze-della-vita 
		
	\ecvitem{-}{Presidente del comitato di valutazione per gara di appalto concessioni di mq 170.000 di cui all'avviso n.57 del 07/06/2019. Committente: Autorità Portuale Mar Tirreno Settentrionale. Periodo Gennaio - Giugno 2020.} 
		%Link: http://portaleservizi.portialtotirreno.it/openweb/pratiche/dett\_registri.php? \\ sezione=avvisiBandi\&id=2497\&codEstr=NEXT % Rif. atti University
		
	\ecvitem{-}{Valutatore Esperto (valutazione in itinere, in loco, finale) di progetti di ricerca e sviluppo Regione Toscana, programma Smart Factory POR FESR 2014-202 D.D. n. 7165 del 24.05.2017,  Bando 1 RS 2017 “Progetti strategici di ricerca e sviluppo”, D.D. n. 7429 del 31.05.2017, Bando 2 RS 2017 “Progetti di ricerca e sviluppo delle MPMI” e D.D. n. 8497 del 05.06.2017, Bando 3 RS 2017 “Bando 3 “Progetti di ricerca e sviluppo attuativi dei Protocolli di Insediamento”. Committente Sviluppo Toscana. Periodo 1 Settembre 2018 - Dicembre 2021.} 
		%Link: https://www.sviluppo.toscana.it/fesr\_14-20 % Rif. atti University
\newpage
  \ecvtitle{2004 -- In corso}{Didattica}
\ecvitem{}{
	\begin{ecvitemize}
		\item[-] Docente esterno / convenzione CNR del corso di ``Supply Chain Management''
		\item[] Data: da A.A. 2011/2012 - in corso
		\item[] Note: Università di Roma ``Tor Vergata'', Dipartimento di Ingegneria dell'Impresa; Corso di Ingegneria Gestionale Dm.270/04 - MAT/09; 6 crediti (CFU);
		Rif. \url{http://didatticaweb.uniroma2.it/docenti/curriculum/3127-Giuseppe-Stecca} 
	\end{ecvitemize}
	
	\begin{ecvitemize}
		\item[-] Docente di ``Modelli di Sistemi di Produzione''
		\item[] Data: da A.A. 2017/2018 a A.A. 2024/2025
		\item[] Note: Università ``Guglielmo Marconi'' Scienze e tecnologie  applicate; corso di Ingegneria Informatica Ing/35; 6 crediti (CFU); Rif. \url{https://www.unimarconi.it/en/docenti}
	\end{ecvitemize}
	
	\begin{ecvitemize}
		\item[-] Docente a contratto per società private sui seguenti temi: Logistica (Formazione finanziata Regione Lazio), Supply Chain e Import / Export (LUISS - Unicredit), Project Management (Education Time), Progettazione Agile (Education Time), Sviluppo software con python o java (Education Time), UML (Education Time)
		\item[] Data: Dal 2004 al 2019
	\end{ecvitemize}
	
	\begin{ecvitemize}\item[-] Supervisore di due studenti di Dottorato, oltre 50 tesi magistrali, 6 assegni di ricerca annuali o collaboratori di ricerca
	\end{ecvitemize}
	
	\begin{ecvitemize}
		\item[-] Membro di commissioni di Laurea Magistrale in Ingegneria Gestionale presso l'Università di Roma ``Tor Vergata''. Dal:  2012 al 06/2025 % Rif. atti University
		
		\item[-] Membro di commissioni di Laurea in Ingegneria Informatica presso Universit\`a ``Guglielmo Marconi''. Dal 2017. % Rif. atti University
		\item[-] Membro di commissione per bandi di assegni di ricerca. Dal 2014. %. Rif. Prot. CNR-IASI 221 del 10/04/2014; Prot. CNR-IASI 935 del 05/12/2016
	\end{ecvitemize}
}
% \ecvitem{}{Working in a research team carrying out in-depth qualitative evaluation of the 2 year Advanced Training of Trainers in Europe using participant observations, in-depth interviews and focus groups. Work carried out in training courses in Strasbourg, Slovenia and Budapest.}

\ecvtitle{2019 - ora}{Attività Editoriali Recenti}
% \ecvitem{}{European Commission, Youth Unit, DG Education and Culture \newline 200, Rue de la Loi, 1049 Brussels (Belgium)}
\ecvitem{}{
	\begin{ecvitemize}
		\item[-] Autore di oltre 80 pubblicazioni scientifiche internazionali di cui 52 indicizzate scopus, authorId=23006716500
		\item[-] Associate Editor per ``Soft Computing''
		\item[-] Associate Editor per ``Open Transportation Journal'
		\item[-] Guest Editor per ``Sustainability''
		\item[-] Guest Editor per ``Discrete Applied Mathematics''
		\item[-] Associate Editor per ``IET Collaborative Intelligent Manufacturing''
		\item[-] Editor Volume ``Graphs and Combinatorial Optimization: from Theory to Applications: CTW2020 Proceedings'', Gentile G, Stecca G., and Ventura P., Springer International Publishing, 2021
		\item[-] Editor Volume ``A View of Operations Research Applications in Italy'', Dell'Amico M., Gaudioso M., and Stecca G., Springer Springer International Publishing, 2019
		
	\end{ecvitemize}
}
  

   \ecvtitle{}{Altre attività recenti, associazioni e nomine}
 \ecvitem{}{
 	\begin{ecvitemize}
 		\item[-] Abilitazione	Scientifica Nazionale alle funzioni di professore universitario di Seconda Fascia nel Settore
 		Concorsuale 01/A6 - RICERCA OPERATIVA
 		\item[] Data: 17/07/2024
 		\item[] Ministero dell'Università e della Ricerca
 		\item[] Note: ASN 2023/2025 N: 22913
 	\end{ecvitemize}
 	\begin{ecvitemize}
 		\item[-] Membro collegio scuola Dottorato ABRO - Automatica Bioingegneria e Ricerca Operativa 
 		\item[] Data: Da A.A. 2021/2022 - in corso
 		\item[] Note: Università di Roma ``La Sapienza'', Dipartimento di Automatica e Ingegneria Gestionale ``Antonio Ruberti'';
 		Rif. \url{https://phd.uniroma1.it/web/AUTOMATICA-BIOINGEGNERIA-E-RICERCA-OPERATIVA---ABRO_nD3553_IT.aspx}
 	\end{ecvitemize}
 	
 	\begin{ecvitemize}
 		\item[-] Membro di AIRO - Associazione Italiana di Ricerca Operativa 
 		\item[] Data: dal 2012 al 2023
 	\end{ecvitemize}
 
  	\begin{ecvitemize}
 	\item[-] Membro dell'Ordine degli Ingegneri della Provincia di Roma 
 	\item[] Data: dal 2004 - in corso
 	\item[] Note: N. 24370, Settori Civile, Ambientale, Industriale, Informazione
 	\end{ecvitemize}
 	
  	\begin{ecvitemize}
	\item[-] Membro della commissione  MANAGEMENT e INGEGNERIA NELLE IMPRESE E NEGLI ENTI, Ordine degli Ingegneri della Provincia di Roma 
	\item[] Data: dal 2023 - in corso
 	\end{ecvitemize}
 	
  	\begin{ecvitemize}
	\item[-] Membro eletto nel Consiglio di Istituto, CNR-IASI
	\item[] Data: dal 2024 - in corso
	\end{ecvitemize}
 }
  
  


  
%  \pagebreak
  
  \ecvsection{Principali competenze}
  \ecvmothertongue{Italiano}
  \ecvlanguageheader
  \ecvlanguage{Inglese}{B2}{B2}{B2}{B2}{C1}
  %\ecvlanguagecertificate{Diplôme d'études en langue française (DELF) B1}
  %\ecvlastlanguage{German}{A2}{A2}{A2}{A2}{A2}
  %\ecvlanguagefooter

  %\
   
  % \ecvblueitem{Communication skills}{
  % \begin{ecvitemize}
  %   \item team work: I have worked in various types of teams from research teams to national league hockey. For 2 years I coached my university hockey team
  %   \item mediating skills: I work on the borders between young people, youth trainers, youth policy and researchers, for example running a 3 day workshop at CoE Symposium ``Youth Actor of Social Change'', and my continued work on youth training programmes 
  %   \item intercultural skills: I am experienced at working in a European dimension such as being a rapporteur at the CoE Budapest ``youth against violence seminar'' and working with refugees.
  % \end{ecvitemize}
  % }
  
  % \ecvblueitem{Organisational / managerial skills}{
  % \begin{ecvitemize}
  %   \item whilst working for a Brussels based refugee NGO ``Convivial'' I organized a ``Civil Dialogue'' between refugees and civil servants at the European Commission 20th June 2002
  %   \item during my PhD I organised a seminar series on research methods
  % \end{ecvitemize}
  % }

  \ecvdigitalcompetence{\ecvProficient}{\ecvProficient}{\ecvProficient}{\ecvProficient}{\ecvProficient}
  
  \ecvblueitem{Competenze Informatiche}{
 MS Office, Libreoffice, Sistemi Operativi Windows and Linux, Programming Languages C/C++ and Python, OpenLCA, SIMIO, JAAMSIM, VISUM.
}

  \ecvblueitem{Competenze Trasferimento Tecnologico}{
	Sviluppati diversi prototipi output di progetti di ricerca in ambito logistica, manufacturing, servizi. https://github.com/gstecca.
}

  \ecvblueitem{Alta Educazione \& Competenze Training}{
	 Organizzatore della scuola di Dottorato MINOA PhD School Mixed-Integer Nonlinear Optimization meets Data Science, in Ischia (Italia) Giugno 25-27, 2019.
}
  \ecvblueitem{Competenze Project Management}{
	Responsabile di pià di 10 Progetti di Ricerca (sia regionali che nazionali) per il CNR. Precedentemente Docente di Project Management e Agile Project Management per aziende private.
}

  \ecvblueitem{Principali Collaborazioni Scientifiche}{
	Università di Roma ``Tor Vergata", Università di Genova, Università di Kobe Giappone, Centro Ricerche FIAT, Università di Palermo, Università di Napoli, Università di Catania. 
}
%  \ecvblueitem{Pubblicazioni}{
%   Autore di circa 50 pubblicazioni su riviste internazionali o convegni internazionali, di cui 30 indicizzate su Scopus. 310 citazioni su google scholar, 166 su scopus. Elenco pubblicazioni su https://scholar.google.it/citations?user=QDZ4kJoAAAAJ\&hl=it
%  }
  
  %\ecvblueitem{Other skills}{Creating pieces of Art and visiting Modern Art galleries. Enjoy all sports particularly hockey, football and running. Love to travel and experience different cultures.}

  %\ecvblueitem{Driving licence}{A, B}

  

  \nocite{*}
  \renewcommand{\section}[2]{\ecvsection{#2}}  \printbibtabular[title=Estratto 20 Pubblicazioni]


   \ecvblueitem{Dichiarazioni} {Autorizzo il trattamento dei dati personali in accordo la legge Italiana ``Decreto Legislativo 30 giugno 2003, n. 196" ``Codice in materia di protezione dei dati personali” e dell’art. 13 del GDPR (Regolamento UE 2016/679).}

\end{europasscv}
  
\end{document}
